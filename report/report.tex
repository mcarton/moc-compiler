\documentclass{scrartcl}
\usepackage[utf8x]{inputenc}
\usepackage[frenchb]{babel}
\usepackage{amssymb}
\usepackage{lmodern}
\usepackage[T1]{fontenc}
\usepackage{hyperref}
\usepackage{minted}

\usepackage{fancyhdr}
\pagestyle{fancy}
\renewcommand{\headrulewidth}{1pt}
\renewcommand{\footrulewidth}{\headrulewidth}

\title{Rapport projet de TDL}
\author{Maxence Ahlouche \and Martin Carton \and Clément Hubin-Andrieu}
\date{mai 2014}

\newcommand{\mocc}{\texttt{mocc}}
\newcommand{\llvm}{\textit{llvm}}
\newcommand{\tam} {\textit{tam}}
\newminted[moccode]{cpp}{%
  tabsize=4, fontsize=\footnotesize,
  frame=lines, framesep=\fboxsep,
  rulecolor=\color{gray!40}
}

\begin{document}
  \maketitle
  \tableofcontents
  \newpage

\section{Introduction}
  Nous avons décider de compiler en \tam{} et en langage intermédiaire \llvm{}.
  De plus les interfaces devraient être suffisamment génériques pour permettre
  de compiler dans n'importe quel langage.

  La compilation en langage intermédiaire \llvm{} nous permet de générer des
  exécutables natifs utilisant la \textit{libc}.

  Nous avons donc supprimé l'assembleur en-ligne tel qu'il était fourni (dans
  \verb+ASM.egg+), et rajouté une instruction \verb+asm+ qui prend une chaine
  de caractères, ce qui permet d'inclure de l'assembleur \llvm{} ou \tam{}
  selon la machine cible voulue.

\section{Tests}
  Nous avons écrit beaucoup de tests. Il y en a 4 types (tous dans le
  dossier \verb+tests+):
  \begin{itemize}
    \item ceux des dossiers \verb+success+, \verb+warning+, \verb+failure+
      testent l'analyse syntaxique et sémantique du compilateur (respectivement
      que le code est correct, qu'il génère un warning ou qu'il génère une
      erreur). Ces exemples ne font rien de particulier;
    \item ceux du dossier \verb+runnable+ testent le code généré, les exemples
      sont compilés par \mocc{} puis doivent être compilés avec \verb+llc+
      ou lancés dans \verb+tam+ et lancés.

      La sortie attendue pour le programme \verb+exemple.moc+ se trouve dans
      le fichier \verb+exemple.moc.output+.

      Ces fichiers nécessitent d'être préprocessés avant d'être compilés afin
      d'y inclure les fonctions d'affichages spécifique à \llvm{} ou \tam{}
      (écrites en assembleur).
  \end{itemize}

\section{Design decisions}
  \subsection{Typage}
    Le typage est plus fort qu'en C.

    En particulier, il n'y a pas de type \verb+void*+, ce qui empêche notamment
    d'écrire une fonction comme \verb+malloc+, mais nous avons ajouté un
    opérateur \verb+new+ (voir section~\ref{new}).

    Le typage des tableaux (voir section~\ref{tab}) est aussi plus fort qu'en C.

  \subsection{Noms de types}
    Les noms de type commencent tous par une majuscule. Ceci afin de permettre
    d'avoir des alias de type (voir section~\ref{alias}) et par uniformité avec
    les nom de classes.

  \subsection{\texttt{NULL} et \texttt{nil}, \texttt{YES} et \texttt{NO}}
    Il n'y a pas de \verb+nil+ (qui serait inutile vu qu'il n'y aurait pas de
    différence avec \verb+NULL+) et \verb+NULL+ s'écrit \verb+null+ par
    consistance avec les autres variables.

    De même \verb+YES+ et \verb+NO+ s'écrivent \verb+yes+ et \verb+no+.

  \subsection{Classes}
    Nous avons supprimé le \verb+@+ devant \verb+@class+. Les méthodes se
    mettent entre les accolades, après les attributs; il n'y a donc plus de
    \verb+@end+.

\section{Extension du langage}
  \subsection{Alias de type}\label{alias}
    Il est possible de définir des alias de type:
    \begin{moccode}
using NouveauNom = NomExistant;
    \end{moccode}

    La syntaxe évite volontairement le \verb+typedef+ bizarre du C.

  \subsection{Tableaux}\label{tab}
    On peut créer des tableaux:
    \begin{moccode}
Char[5] s = "net7";
    \end{moccode}

    La taille se met après le type, et non le nom comme en C.

    Ils sont convertibles en pointeurs vers le type correspondant, mais le font
    dans moins de cas qu'en C (notamment les tableaux sont copiables,
    passables comme paramètre de fonction et peuvent être retournés). Le type
    tableau est un vrai type.

  \subsection{Opérateurs new et delete}\label{new}
    Ces opérateurs sont équivalents à \verb+malloc+ et \verb+free+, mais sont
    typés (le langage ne possède pas de type \verb+void*+).

    \begin{moccode}
Int* taille = new(Int);
*taille = 10;

Char* test = new(*taille, Char);
delete(taille);
delete(test);
    \end{moccode}

  \subsection{Opérateur sizeof}
    Cet opérateur retourne la taille d'un type:

    \begin{moccode}
Int a = sizeof(Int); // 1 en tam, 8 en llvm sur une machine 64bits
    \end{moccode}

  \subsection{Boucle}
    Nous avons ajouté une boucle \verb+while+.

\section{Warnings}
  Nous avons ajouté des warnings au compilateur.

  Il suffit d'appeler \mocc{} avec \verb+-w nom_du_warning+.
  Il y a aussi un warning \verb+all+.

  \subsection{unreachable}
    \verb"unreachable" vérifie la présence d'instructions inutiles.

    Par exemple:
    \begin{moccode}
Int test() {
    [...]

    if(a) {
        return 123;
    }
    else {
        return 456;
    }
    f(); // unreachable
}
    \end{moccode}

  \subsection{shadow}
    \verb"shadow" vérifie qu'une déclaration ne masque pas une déclaration
    précédente, par exemple:
    \begin{moccode}
Int test() {
    Int a;

    if(a) {
        Char a; // shadow
    }
}
    \end{moccode}
\end{document}

