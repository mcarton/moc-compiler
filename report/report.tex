\documentclass{scrartcl}
\usepackage[utf8x]{inputenc}
\usepackage[frenchb]{babel}
\usepackage{amssymb}
\usepackage{lmodern}
\usepackage[T1]{fontenc}
\usepackage{hyperref}
\usepackage{minted}

\usepackage{fancyhdr}
\pagestyle{fancy}
\renewcommand{\headrulewidth}{1pt}
\renewcommand{\footrulewidth}{\headrulewidth}

\title{Rapport projet de TDL}
\author{Maxence Ahlouche \and Martin Carton \and Clément Hubin-Andrieu}
\date{mai 2014}

\newcommand{\mocc}{\textit{mocc}}
\newminted[moccode]{cpp}{%
  tabsize=4, fontsize=\footnotesize,
  frame=lines, framesep=\fboxsep,
  rulecolor=\color{gray!40}
}

\begin{document}
  \maketitle
  \tableofcontents
  \newpage

\section{Design decisions}
  \subsection{Typage}
    Le typage est plus fort qu'en C.  Ça empêche notamment d'écrire une
    fonction comme \verb+malloc+, mais on a un opérateur \verb+new+ (voir
    section~\ref{new}) qui fait ça très bien.

  \subsection{Noms de types}
    Les noms de type commencent tous par une majuscule.
    Ceci afin de permettre d'avoir des alias de type (voir
    section~\ref{alias}).

  \section{NULL et nil, YES et NO}
    Il n'y a pas de \verb+nil+ (qui serait inutile vu qu'il n'y aurait pas de
    différence avec \verb+NULL+) et \verb+NULL+ s'écrit \verb+null+ par
    consistance avec les autres variables.

    De même \verb+YES+ et \verb+NO+ s'écrivent \verb+yes+ et \verb+no+.

  \section{Classes}
    Nous avons supprimé les \verb+@+ devant \verb+@class+ et \verb+@end+.

\section{Extension du langage}
  \subsection{Alias de type}\label{alias}
    Il est possible de définir des alias de type:
    \begin{moccode}
using NouveauNom = NomExistant;
    \end{moccode}

    La syntaxe évite volontairement le \verb+typedef+ bizarre du C.

  \subsection{Tableaux}
    On peut créer des tableaux:
    \begin{moccode}
Char[5] s = "net7";
    \end{moccode}

    La taille se met après le type, et non le nom comme en C.

    Ils sont convertibles en pointeurs vers le type correspondant, mais le font
    dans moins de cas qu'en C (notamment les tableaux sont copiables et
    passables comme paramètre de fonction et peuvent être retournés). Le type
    tableau est un vrai type.

  \subsection{Opérateurs new et delete}\label{new}
    Ces opérateurs sont équivalents à \verb+malloc+ et \verb+free+, mais sont
    typés (le langage ne possède pas de type \verb+void*+).

    \begin{moccode}
Int* taille = new(Int);
*taille = 10;

Char* test = new(*taille, Char);
delete(taille);
delete(test);
    \end{moccode}

  \subsection{Opérateur sizeof}
    Cet opérateur retourne la taille d'un type:

    \begin{moccode}
Int a = sizeof(Int);
    \end{moccode}

\section{Warnings}
  Nous avons ajoutés des warnings au compilateur.

  Il suffit d'appeler \mocc{} avec \verb+-w nom_du_warning+.
  Il y a aussi un warning \verb+all+.

  \subsection{unreachable}
    \verb"unreachable" vérifie la présence d'instructions inutiles.

    Par exemple:
    \begin{moccode}
Int test() {
    if(a) {
        return 123;
    }
    else {
        return 456;
    }
    f(); // unreachable
}
    \end{moccode}

  \subsection{shadow}
    \verb"shadow" vérifie qu'une déclaration ne masque pas une déclaration
    précédente.
\end{document}

