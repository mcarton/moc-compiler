\documentclass{scrartcl}
\usepackage[utf8x]{inputenc}
\usepackage[frenchb]{babel}
\usepackage{amssymb}
\usepackage{lmodern}
\usepackage[T1]{fontenc}
\usepackage{hyperref}
\usepackage{minted}

\usepackage{fancyhdr}
\pagestyle{fancy}
\renewcommand{\headrulewidth}{1pt}
\renewcommand{\footrulewidth}{\headrulewidth}

\title{Rapport projet de TDL}
\author{Maxence Ahlouche \and Martin Carton \and Clément Hubin-Andrieu}
\date{4 mai 2014}

\newcommand{\egg} {\texttt{egg}}
\newcommand{\mocc}{\texttt{mocc}}
\newcommand{\llvm}{\textit{llvm}}
\newcommand{\tam} {\textit{tam}}
\newcommand{\objc}{\textit{Objective-C}}
\newminted[moccode]{cpp}{%
  tabsize=4, fontsize=\footnotesize,
  frame=lines, framesep=\fboxsep,
  rulecolor=\color{gray!40}
}

\begin{document}
  \maketitle
  \tableofcontents
  \newpage

\section{Introduction}
  Nous avons décider de compiler en \tam{} et en langage intermédiaire \llvm{}.
  De plus les interfaces devraient être suffisamment génériques pour permettre
  de compiler dans n'importe quel langage. Pour utiliser la machine \llvm{}, il
  suffit d'appeler le script \texttt{mocc} avec l'option \verb+-m llvm+ (la
  machine \tam{} est la machine par défaut).

  La compilation en langage intermédiaire \llvm{} nous permet de générer des
  exécutables natifs utilisant la \textit{libc} bien plus facilement, et de
  manière plus portable\footnote{Bien que nous compilions uniquement pour une
  machine linux 64~bits, le porter en 32~bits serait assez simple. Le porter
  sur Windows nécessiterait uniquement d'ajouter des déclarations pour le
  linkage.} que ne l'aurait été la génération en directement en
  \textit{x86}. De plus le langage étant plus expressif (il est notamment typé),
  il permet de trouver plus facilement les erreurs éventuelles. Enfin, il nous
  permet de bénéficier des différentes passes d'optimisation du compilateur
  \verb+llc+ (il est intéressant de noter qu'il est possible de voir le
  résultat après chaque passe).

  Nous avons donc supprimé l'assembleur en-ligne tel qu'il était fourni (dans
  \verb+ASM.egg+), et rajouté une instruction \verb+asm+ qui prend une chaine
  de caractères, ce qui permet d'inclure de l'assembleur \llvm{} ou \tam{}
  selon la machine cible voulue.

  \paragraph{}
  \tam{} a cependant plusieurs avantages par rapport à \llvm{}: pour commencer,
  \llvm{} est beaucoup plus expressif et typé: c'est souvent un avantage, mais
  rend parfois certaines tâches très complexe (par exemple la gestion de la
  \textit{vtable}, il est aussi nécessaire de préciser le type des variables et
  registres à chaque utilisation (et non déclaration), ce qui est très
  répétitif).

  Ensuite le nombre d'instructions très réduit de \tam{} rend la machine \tam{}
  relativement simple, alors que la machine \llvm{} est bien plus complexe.

  Enfin la possibilité de voir l'évolution de la pile et du tas instruction par
  instruction est un gros avantage de \tam{}. Pour pouvoir débugger la machine
  \llvm{} il aurait fallu ajouter les informations de débuggage au fichier
  construit, rendant la machine encore plus complexe.

\section{Tests}
  Nous avons écrit beaucoup de tests. Il y en a 4 types (tous dans le
  dossier \verb+tests+):
  \begin{itemize}
    \item ceux des dossiers \verb+success+, \verb+warning+, \verb+failure+
      testent l'analyse syntaxique et sémantique du compilateur (respectivement
      que le code est correct, qu'il génère un warning (voir section
      \ref{sec:warnings}) ou qu'il génère une erreur). Ces exemples ne font
      rien de particulier;
    \item ceux du dossier \verb+runnable+ testent le code généré, les exemples
      sont compilés par \mocc{} puis doivent être compilés avec \verb+llc+
      ou lancés dans \verb+tam+ et lancés.

      La sortie attendue pour le programme \verb+exemple.moc+ se trouve dans
      le fichier \verb+exemple.moc.output+.

      Ces fichiers nécessitent d'être préprocessés avant d'être compilés afin
      d'y inclure les fonctions d'affichages spécifique à \llvm{} ou \tam{}
      (écrites en assembleur).

      Les scripts \verb+tests/run-llvm+ et \verb+tests/run-tam+ permettent de
      préprocesser, compiler et lancer ces tests.
  \end{itemize}

\section{Génération de code}
  Nous avons utilisé la syntaxe de Maxime, celle-ci étant plus légère (elle
  permet notamment de chainer les appels de méthode).

  \paragraph{}
  Il n'y a pas de code spécifique à \llvm{} ou à \tam{} en dehors de leur
  dossier respectif. L'interface \verb+IMachine+ et le \egg{} ont été prévus
  pour être le plus génériques possible, on devrait pas exemple pouvoir
  l'utiliser pour générer du \textit{C}.

  En particulier, la classe \verb+Type+ n'a pas de taille. En effet, si l'on
  devait générer par exemple du \textit{C}, celle-ci ne servirait pas (on
  utiliserait plutôt l'opérateur \verb+sizeof+). Pour récupérer la taille d'un
  type, mais aussi son nom en \llvm{} on utilise un visiteur (par exemple le
  type \verb+Int+ est représenté par un \verb+i64+, la constante \verb+"hello"+
  par un \verb+[6 x i8]+ -- le nom n'ayant de sens qu'avec cette machine).

  \paragraph{}
  De même, il y a deux interfaces \verb+IExpr+ et \verb+ILocation+ utilisées
  par l'interface \verb+IMachine+ et le \egg{}. Ces interfaces représentent
  une expression et un emplacement mémoire pour chaque machine.

  En \llvm{}, une expression contient soit un emplacement mémoire (qui est un
  nom de la forme \verb+%nom+ pour les variables nommées, ou \verb+%5+ pour les
  temporaires) et le code nécessaire pour la générer, soit une constante (par
  exemple \verb+42+, ou encore \verb+null+). Une expression contient aussi un
  booléen indiquant s'il est nécessaire d'utiliser l'instruction \verb+load+
  pour obtenir la valeur.

  En \tam{}, une expression contient le code nécessaire pour la générer, et un
  booléen indiquant s'il est nécessaire d'utiliser l'instruction \verb+loadi+
  pour obtenir la valeur (ce booléen n'a pas forcément la même valeur que
  celui de la machine \llvm{}, et serait inutile si l'on générait par exemple
  du \textit{C}, il ne fait donc pas partie de l'interface commune). Les
  expressions \tam{} n'ont pas besoin de contenir d'adresse, celles-ci ne
  servent que pour obtenir l'expression associée à un identifiant de la table
  des symboles (variable locale, paramètre de fonction, attribut d'une classe).

\section{Extensions du langage}
  \subsection{Asm}
    Comme indiqué dans l'introduction, nous avons remplacé l'instruction
    \verb+asm+ fournie par une instruction plus simple prenant une chaine de
    caractère.

    \begin{moccode}
// en llvm:
void put_char(Char c) {
    asm("%1 = load i8* %c.0, align 1");
    asm("%2 = sext i8 %1 to i32");
    asm("%3 = call i32 @putchar(i32 %2)");
}

// en tam:
void put_char(Char c) {
    asm("LOAD (1) -1[LB]");
    asm("SUBR COut");
}
    \end{moccode}

    Il n'est pas possible d'utiliser les variables autrement que par leur
    adresse. Vu le peu d'assembleur en-ligne que nous utilisons, nous n'avons
    pas jugé qu'ajouter la possibilité d'utiliser des variables par leur nom
    vaille la peine d'être développé.

  \subsection{Alias de type}\label{sec:alias}
    Il est possible de définir des alias de type:
    \begin{moccode}
using NouveauNom = NomExistant;
    \end{moccode}

    La syntaxe évite volontairement le \verb+typedef+ bizarre du C.

  \subsection{Tableaux}\label{sec:tab}
    On peut créer des tableaux:
    \begin{moccode}
Char[5] s = "net7";
    \end{moccode}

    La taille se met après le type, et non le nom comme en C.

    Ils sont convertibles en pointeurs vers le type correspondant, mais le font
    dans moins de cas qu'en C (notamment les tableaux sont copiables,
    passables comme paramètre de fonction et peuvent être retournés). Le type
    tableau est un vrai type.

  \subsection{Opérateurs new et delete}\label{sec:new}
    Ces opérateurs sont équivalents à \verb+malloc+ et \verb+free+, mais sont
    typés (le langage ne possède pas de type \verb+void*+).

    \begin{moccode}
Int* taille = new(Int);
*taille = 10;

Char* test = new[*taille](Char);
delete(taille);
delete(test);
    \end{moccode}

    L'opérateur \verb+new+ a aussi l'avantage de tenir compte de la taille et
    d'initialiser la \textit{vtable} (voir section \ref{sec:vtable}) dans le
    cas d'une classe.

  \subsection{Opérateur sizeof}
    Cet opérateur retourne la taille d'un type:

    \begin{moccode}
Int a = sizeof(Int); // 1 en tam, 8 en llvm sur une machine 64bits
    \end{moccode}

    Il tient compte de la \textit{vtable} dans le cas d'une classe.

  \subsection{Boucle}
    Nous avons ajouté une boucle \verb+while+.

\section{Design decisions}
  \subsection{Typage}
    Le typage est plus fort qu'en C.

    En particulier, il n'y a pas de type \verb+void*+, ce qui empêche notamment
    d'écrire une fonction comme \verb+malloc+, mais nous avons ajouté un
    opérateur \verb+new+ (voir section~\ref{sec:new}).

    Le typage des tableaux (voir section~\ref{sec:tab}) est aussi plus fort qu'en C.

    Cependant les entiers se comportent comme des booléens: nous avons commencé
    comme cela, nous aurions dû ajouter le type booléen dès le début.

    \subsubsection{Conversions}
      Il a deux types de casts: implicites et explicites.

      Les casts implicites ont lieux dans peu de cas. Par exemple la constante
      \verb+null+ est de type \verb+NullType+ mais est convertible
      implicitement vers tout les types de pointeur et un tableau est
      convertible vers un pointeur du même type. De même, dans un \verb+if+ ou
      un \verb+while+ la «~condition~» peut être un pointeur.

      Les casts explicites se font à l'aide de \verb+(Type)valeur+. Tout les
      casts ne sont pas permis cependant: par exemple un caractère ne peut pas
      être converti en pointeur, bien qu'en \tam{} ils soient tout les deux
      représentés par un mot et qu'en \llvm{} un caractère fait 8 bits et un
      pointeur 64.

  \subsection{Noms de types}
    Les noms de type commencent tous par une majuscule. Ceci afin de permettre
    d'avoir des alias de type (voir section~\ref{sec:alias}) et par uniformité avec
    les nom de classes.

    Cette restriction permet de désambiguïser une instruction comme
    \verb+a*b;+ qui pourrait être interprétée comme la multiplication de
    \verb+a+ et \verb+b+ ou la déclaration d'un pointeur \verb+b+ de type
    \verb+a*+.

  \subsection{\texttt{NULL} et \texttt{nil}, \texttt{YES} et \texttt{NO}}
    Il n'y a pas de \verb+nil+ (qui serait inutile vu que nous n'avons pas vu
    de différence avec \verb+NULL+) et \verb+NULL+ s'écrit \verb+null+ par
    consistance avec les autres variables.

    De même \verb+YES+ et \verb+NO+ s'écrivent \verb+yes+ et \verb+no+.

  \subsection{Classes}
    Nous avons supprimé le \verb+@+ devant \verb+@class+. Les méthodes se
    mettent entre les accolades, après les attributs; il n'y a donc plus de
    \verb+@end+.

    Par exemple:
    \begin{moccode}
class Point {
    Int x;
    Int y;

    +(void) init {
    }

    -(Int) x {
        return 0;
    }

    -(Int) y {
        return 0;
    }
}
    \end{moccode}

    \subsubsection{Table virtuelle}\label{sec:vtable}
      Afin de permettre l'appel de méthode par liaison tardive, chaque classe
      possède un membre implicite: sa table virtuelle. Ce membre est
      implicitement initialisé par l'opérateur \verb+new+.

      Lors d'un appel de méthode sur une instance de type connu, l'existence de
      la méthode et le type des paramètres est vérifié à la compilation. La
      méthode à appeler est cherchée dans la \textit{vtable} par son index à
      l'exécution.

      \paragraph{}
      Cependant, afin de permettre d'appeler une méthode sur une instance de
      type inconnu (\verb+id+) la table virtuelle n'est pas un simple tableau
      contenant l'adresse des fonctions. À la place, c'est une table
      associative \verb+nom -> adresse+ (où le «~nom~» est par exemple de la
      forme \verb+x:y:z:+ pour refléter le fait qu'en \objc{} une méthode peut
      avoir plusieurs noms).

      Pour appeler une méthode sur une instance de type inconnu, on pourrait
      alors chercher la méthode, par son nom dans cette table. Cependant, nous
      n'avons pas eu le temps d'implémenter ce type d'appel.

\section{Warnings}\label{sec:warnings}
  Nous avons ajouté des warnings au compilateur.

  Il suffit d'appeler \mocc{} avec \verb+-w nom_du_warning+.
  Il y a aussi un warning \verb+all+.

  \subsection{unreachable}
    \verb"unreachable" vérifie la présence d'instructions inutiles.

    Par exemple:
    \begin{moccode}
Int test() {
    [...]

    if(a) {
        return 123;
    }
    else {
        return 456;
    }
    f(); // unreachable
}
    \end{moccode}

  \subsection{shadow}
    \verb"shadow" vérifie qu'une déclaration ne masque pas une déclaration
    précédente, par exemple:
    \begin{moccode}
Int test() {
    Int a;

    if(a) {
        Char a; // shadow
    }
}
    \end{moccode}
\end{document}

